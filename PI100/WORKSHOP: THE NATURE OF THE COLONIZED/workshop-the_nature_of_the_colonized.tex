\documentclass[12pt]{article}

% packages
\usepackage{graphicx}
\usepackage{hyperref}
\usepackage{amsmath}
\usepackage{amssymb}
\usepackage{natbib}
\usepackage{apalike}

% title
\title{Cuestionario 2: The Nature of the Colonized}

% include authors file
\input{authors.authors}

\date{2023-04-28}




\begin{document}

% title page
\maketitle

\newpage

% table of contents
\tableofcontents

\newpage

% abstract (optional)
\begin{abstract}
Provide the literature on the political, economic, and cultural features prescribed in the fragebogen. 
\end{abstract}





\section{Ancient Philippine Society}
  \subsection{Location of the Philippines}
  The Philippines is a country located in Southeast Asia. It is a part of the nations in Southeast Asia forming a fully opened fan \citep{corpuz1965philippines}. 
  It is in the south of Taiwan, north of Indonesia, and east of Vietnam. The islands are located between bodies of water such as the South China Sea 
  on the left and the Pacific Ocean on the right. In the map, the country is situated at \textbf{14.5995° N} latitude and \textbf{120.9842° E} longitude.
  \subsection{Prominent Features of the Philippines}
  as
  \subsection{Philippines' Encounter with Spain}
  asd
  \subsection{The Balangay}
  sdf






\section{Political Institutions Introduced by Spain}
Insert background information here.






\section{Economic Institutions Introduced by Spain}
Insert methodology here.





\section{Cultural Institutions Introduced by Spain}
Insert results here.






% references
\bibliographystyle{apalike}
\bibliography{references}

\end{document}
